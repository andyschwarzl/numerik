\section{Approximation von Funktionen}
% hat er es nur umbenannt?
\section{Interpolation}
Aufgabenstellung: Aus einer festgelegten Menge von Funktionen Mn 
bestimme man eine Funktion, die durch die gegebenen Punkte
$(x_0, f_0), (x_1, f_1), \cdots, (x_n, f_n) \in \mathbb{R}^2$ verläuft.

\missingfigure{Funktion mit Stützstellen}
Die Wahl von Mn ist abhängig von der Problemstellung:
\begin{itemize}
  \item $\Pi_n$: Menge der Polynome mit Grad $\leq$ n
  \item stückweise polynomiale Funktion
  \item trigonometrische Funktion
\end{itemize}
Warum und weshalb:
\begin{itemize}
  \item Berechnung von Zwischenwerten einer Funktion, die nur an wenigen 
    Stellen bekannt ist
  \item Vereinfachung der Komplexität einer Funktion. Beschreibung
    einer Funktion durch eine kleine Anzahl von Funktionen; $\Rightarrow$
    einfacheres Rechnen
  \item wichtige theoretische Grundlage für verschiedene andere numerische
    Aufgaben (Integration, Differenzialgleichungen)
\end{itemize}
